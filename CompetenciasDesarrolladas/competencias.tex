\section{Competencias Desarrolladas}

Las siguientes son habilidades o conocimientos aprendidos o reforzados al aplicarse en la elaboración de este proyecto:
\begin{enumerate}
	\item \textbf{Java:} El proyecto esta hecho en java por lo que el nuevo módulo se encuentra programado en su totalidad en java, utilizando sus características de orientación a objetos,  herencia, polimorfismo, interfaces, etc. y además implementando patrones de diseño como \textit{Modelo Vista Controlador} y \textit{Factory}.
	
	\item \textbf{Maven:} Se utilizó la herramienta \textit{Maven} para el manejo de las dependencias dentro del proyecto y para construir y ejecutar el servidor PowerTAC.

	\item \textbf{Manejo de sistema de control de versiones git:} El proyecto se albergó en un repositorio git para que la programación en equipo fuera más ágil.
	
	\item \textbf{Uso de métodos de regresión y clasificación:} Se estudiaron las características de varios algoritmos tanto de regresión como de clasificación, se implementaron usando la librería weka, se compararon y se analizó los pros y contras de cada uno para cada uno de las situaciones que se presentaron.

	\item \textbf{Uso de librería weka:} Se utilizó extensamente para la implementación de los algoritmos usados para hacer las predicciones.

	\item \textbf{Sistema de composición de textos \LaTeX{}:} Se utilizó este sistema en conjunto con el IDE \textit{TexMaker} para la elaboración del reporte técnico y los reportes de residencias incluyendo este documento.
	
	\item \textbf{Trabajo en equipo}: La elaboración de este proyecto requirió una coordinación con otros miembros del equipo de desarrollo.
	
	\item \textbf{Investigación y aprendizaje autónomo}: Gran parte del proceso de desarrollo consistió en investigar conceptos, reglas de la competencia Power TAC, algoritmos de regresión y clasificación, etc. La mayoría de esto de manera autónoma.
	
	\item \textbf{Capacidad de análisis y abstracción}: Estas dos habilidades son necesarias en la resolución de problemas de este tipo ya que se requiere para realizar actividades como: comprender los problemas planteados en la competencia Power TAC, enfocarse en las reglas importantes para la creación del módulo de predicción, reducir los problemas grandes en tareas pequeñas, investigar las herramientas disponibles y abstraer las mas útiles para resolver el problema, leer código que ya esta hecho, comprender su funcionalidad y abstraer lo que es mas importante para la tarea que se esta realizando, entre otras actividades.
	
	\item \textbf{Estructuras de datos}: El desarrollo de este proyecto incluyo el uso de estructuras de datos como los hash set y hash map, arreglos dinámicos, pilas y colas.
	
	\item \textbf{Conceptos de aprendizaje automático}: Los algoritmos de regresión y clasificación usados en el proyecto son clasificados como algoritmos de aprendizaje automático o \textit{machine learning}, por lo que se necesito comprender algunos conceptos sobre este tema.

\end{enumerate}
