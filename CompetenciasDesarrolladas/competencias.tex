\section{Competencias Desarrolladas.}

Las siguientes son habilidades o conocimientos aprendidos o reforzados al aplicarse en la elaboración de este proyecto:
\begin{enumerate}
	\item \textbf{Java:} El proyecto esta hecho en java por lo que el nuevo modulo se encuentra programado en su totalidad en java, utilizando sus características de orientación a objetos,  herencia, polimorfismo, interfaces, etc. y ademas implementando patrones de diseño como \textit{Modelo Vista Controlador} y \textit{Factory}.
	
	\item \textbf{Maven:} Se utilizo la herramienta \textit{Maven} para el manejo de las dependencias dentro del proyecto y para construir y ejecutar el servidor PowerTAC.

	\item \textbf{Manejo de sistema de control de versiones git:} El proyecto se albergo en un repositorio git para que la programación en equipo fuera mas ágil.
	
	\item \textbf{Uso de métodos de regresión y clasificación:} Se estudiaron las características de varios algoritmos tanto de regresión como de clasificación, se implementaron usando la librería weka, se compararon y se analizo los pros y contras de cada uno para cada uno de las situaciones que se presentaron.

	\item \textbf{Uso de librería weka:} Se utilizo extensamente para la implementación de los algoritmos usados para hacer las predicciones.

	\item \textbf{Sistema de composición de textos \LaTeX{}:} Se utilizo este sistema en conjunto con el IDE \textit{TexMaker} para la elaboración del reporte técnico y los reportes de residencias incluyendo este documento.

\end{enumerate}
