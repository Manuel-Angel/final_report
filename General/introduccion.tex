\chapter{Generalidades del Proyecto.}
\newpage
\section*{Introducción}
\addcontentsline{toc}{section}{Introducción}
El proyecto ``CEMIE - Eólico P12: Desarrollo de tecnología basada en inteligencia artificial y mecatrónica, para integrar un parque de generación de energía eólica a una red inteligente'' es un proyecto integrador desarrollado por el INAOE (Instituto Nacional de Astrofísica, Óptica y Electrónica) que abarca diferentes áreas de estudio, tiene como objetivo principal la integración de un parque eólico con otras fuentes de generación, unidades de almacenamiento y consumidores usando un
sistema microgrid.

Le creciente demanda de energía para mantener nuestro estilo de vida y la amenaza de terminar las actuales fuentes de energía no renovables han llevado a la búsqueda de fuentes de energía renovables, y nuevas maneras de aprovechar mejor estas fuentes de energía a través de una red eléctrica inteligente y un mercado energético dinámico que se adapte a la oferta y la demanda.
El uso de energías renovables que se esta dando cada vez más requiere que la energía sea distribuida a través de una red inteligente, es decir una red que permita a consumidores comunes también ser productores (teniendo paneles solares, vehículos eléctricos u otros dispositivos) como tambien pequeñas empresas productoras que exploten las energías renovables.

Junto con la adopción de redes energéticas más inteligentes, surge la idea de regular la oferta y demanda de energía en un mercado libre de una manera más inteligente, donde todo aquel capas de producir energía pueda venderla, esto supone una complejidad mucho mayor al mercado energético clásico donde unos pocos grandes productores venden la energía a todos los consumidores, y hace que sea más complicado que cada productor le cobre directamente a los que consumen su energía, por lo que surge la idea de que existan intermediarios. A estos intermediarios del mercado energético se les denominan \textit{brokers}

Uno de los mercados energéticos dominantes es el de tarifas en donde pequeños consumidores o productores pueden comprar o vender energía de los agentes ``broker'' a través de tarifas, este mercado esta regido totalmente por la oferta y la demanda.
Las tarifas son contratos entre un productor y un broker o un consumidor y un broker, es decir, el broker es un agente intermediario el cual llega a un acuerdo con los productores para vender la energía producida a los consumidores, y este organiza ambas partes para negociar una cierta cantidad de energía bajo ciertas condiciones \cite{MPAlonsoAYRGonzalezDesarrolloDeTec}.

El módulo en donde se llevará a cabo el proyecto ``Agente Inteligente de compra venta de energía'' como parte de las residencias profesionales es en la creación de un nuevo módulo de predicción para el broker (que es un agente intermediario en operaciones financieras o comerciales que percibe una comisión por su intervención \cite{RAEDiccionario2014}) 
COLDPower (que ya dispone el INAOE), para el concurso PowerTAC, el cual se lleva a cabo en una plataforma de simulación que emula un mercado energético, haciendo uso de datos meteorológicos históricos del mundo real e incluyendo modelos de clientes que simulan el comportamiento de diferentes tipos de consumidores y generadores entre otros parámetros. 

Esto ayuda a adelantarnos en las estrategias de compra venta de energía y en estrategias para balancear la red (que la energía producida sea igual a la energía consumida) en escenario que aún no es posible en México por la falta de infraestructura, pero que ya empieza a ser realidad en otras partes del mundo como los países que actualmente organizan y mantiene el concurso Power TAC, los cuales vieron una necesidad en empezar a desarrollar estas técnicas de mercadeo antes de pasar completamente a una infraestructura de red inteligente.