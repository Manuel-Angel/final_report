\chapter{Generalidades del Proyecto.}
\newpage
\begin{center}
    \section*{Introducción}
	El proyecto ``CEMIE - Eólico P12: Desarrollo de tecnología basada en inteligencia artificial y mecatrónica, para integrar un parque de generación de energía eólica a una red inteligente'' es un proyecto integrador desarrollado por el INAOE (Instituto Nacional de Astrofísica, Óptica y Electrónica) que abarca diferentes áreas de estudio, tiene como objetivo principal la integración de un parque eólico con otras fuentes de generación, unidades de almacenamiento y consumidores usando un
sistema microgrid.


El módulo en donde se llevará a cabo el proyecto ``Agente Inteligente de compra venta de energía'' como parte de las residencias profesionales es en la creación de un nuevo módulo de predicción para el broker (que es un agente intermediario en operaciones financieras o comerciales que percibe una comisión por su intervención \cite{RAEDiccionario2014}) 
COLDPower (que ya dispone el INAOE), para el concurso PowerTAC, el cual se lleva a cabo en una plataforma de simulación que emula un mercado energético, haciendo uso de datos meteorológicos históricos del mundo real e incluyendo modelos de clientes que simulan el comportamiento de diferentes tipos de consumidores y generadores entre otros parámetros.


Junto con la adopción de redes energéticas más inteligentes, surge la idea de regular la oferta y demanda de energía en un mercado libre, donde los productores puedan vender energía a consumidores usando un broker como intermediario \cite{Fixed-priceTariffG2015}.
Uno de los mercados energéticos dominantes es el de tarifas en donde pequeños consumidores pueden comprar energía de los agentes ``broker'' a través de tarifas.
Las tarifas son contratos entre un productor y un broker o un consumidor y un broker, es decir, el broker es un agente intermediario el cual llega a un acuerdo con los productores para vender la energía producida a los consumidores, y este organiza ambas partes para negociar una cierta cantidad de energía bajo ciertas condiciones \cite{MPAlonsoAYRGonzalezDesarrolloDeTec}.
\end{center}
\addcontentsline{toc}{section}{Introducción}